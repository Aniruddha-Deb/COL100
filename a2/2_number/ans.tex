\documentclass{article}
\usepackage{amsmath}
\usepackage{mathtools}
\usepackage[a4paper,left=1.5cm,right=1.5cm,top=1cm,bottom=2cm]{geometry}

\newcounter{problem}
\newcounter{solution}

\newcommand\Problem{%
  \addtocounter{problem}{1}%
  \textbf{\alph{problem})}~%
  \setcounter{solution}{0}%
}

\newcommand\TheSolution{%
  \textbf{Solution:}\\%
}

\newcommand\ASolution{%
  \stepcounter{solution}%
  \textbf{Solution \thesolution:}\\%
}

\title{Number System Conversions \vspace{-0.5cm}}
\author{Aniruddha Deb}
\date{\vspace{-1cm}}

\parindent 0in
\parskip 1em
\begin{document}
\maketitle

\Problem $(1001101010)_{2}$ to Decimal

\TheSolution Converting from Binary to Decimal involves multiplying each binary
digit with it's place value. Place values in binary are in powers of 2, so
\begin{align*}
D &= 1 \times 2^{9} + 1 \times 2^6 + 1 \times 2^5 + 1 \times 2^3 + 1 \times 2^1 \\
\Aboxed{D &= 618}
\end{align*}

\Problem $(490)_{10}$ to Octal

\TheSolution Dividing by 8 and reading the remainders upwards gives us the Octal
number
\begin{center}
\begin{tabular}{c|c|c}
8 & 490 & 2 \\
\hline
8 & 61 & 5 \\
\hline
8 & 7 & 7 \\
\hline
 & 0 &
\end{tabular}
\end{center}

Hence, $\boxed{(490)_{10} = (752)_8}$

\Problem $(576)_8$ to Hexadecimal

\TheSolution An easy way to convert between two number systems where the place
values are powers of 2 is to first convert to binary, then group the numbers
together. This works, because the place values being powers of 2, play along
nicely.
\begin{align*}
(576)_8 &= (\underline{101}\,\underline{111}\,\underline{110})_2 \\
(\underline{1}\,\underline{0111}\,\underline{1110})_2 &= (17\mathrm{E})_{16}
\end{align*}
Hence, $\boxed{(576)_{8} = (17\mathrm{E})_{16}}$

\Problem $(\mathrm{B9C0})_{16}$ to Binary

\TheSolution simple in-place conversion of individual hex digits to binary works,
since the base (16) is a power of 2.
\begin{align*}
\Aboxed{(\mathrm{B9C0})_{16} &= (\underline{1011}\,\underline{1001}\,\underline{1100}\,\underline{0000})_2}
\end{align*}

\Problem $(6537)_8$ to Binary

\TheSolution proceed similar to the above:
\begin{align*}
\Aboxed{(\mathrm{6537})_{8} &= (\underline{110}\,\underline{101}\,\underline{011}\,\underline{111})_2}
\end{align*}

\pagebreak

\Problem $(445)_{10}$ to Octal

\TheSolution proceed similar to problem \textbf{b}
\begin{center}
\begin{tabular}{c|c|c}
8 & 445 & 5 \\
\hline
8 & 55 & 7 \\
\hline
8 & 6 & 6 \\
\hline
 & 0 &
\end{tabular}
\end{center}

Hence, $\boxed{(445)_{10} = (675)_8}$

\Problem $(11001)_2$ to Decimal

\TheSolution proceed similar to problem \textbf{a}
\begin{align*}
D &= 1 \times 2^{4} + 1 \times 2^3 + 1 \times 2^0 \\
\Aboxed{D &= 25}
\end{align*}

\Problem $\mathrm{(4AD)}_{16}$ to Decimal

\TheSolution This is similar to converting from Binary to Decimal: take the 
face value of each digit (in base 10) and multiply that with the place value
of that digit
\begin{align*}
D &= 4 \times 16^{2} + 10 \times 16^1 + 13 \times 16^0 \\
\Aboxed{D &= 1197}
\end{align*}

\end{document}


